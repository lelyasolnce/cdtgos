\documentclass[a4paper,10pt]{article}

\usepackage{ifpdf}
\ifpdf
    \usepackage{cmap}
\fi

\usepackage[utf8]{inputenc}
\usepackage[T2A]{fontenc}
\usepackage[russian]{babel}
\usepackage[left=2cm,right=2cm,top=2cm,bottom=2cm]{geometry}

\usepackage{amsmath}


\begin{document}
\section*{Билет \No 2.24}
{\em Шифрование с открытым ключом. Алгоритмы RSA и Эль-Гамаля.
Методы распределение ключей. Алгоритмы разделения секрета.}

\section{Шифрование с открытым ключом} 
{\bf Криптографическая система с открытым ключом} (или {\bf Асимметричное шифрование, Асимметричный шифр})~--- система шифрования 
в которой открытый ключ передаётся по открытому (то есть незащищённому, доступному для наблюдения) каналу, и
используется только для шифрования сообщения. Для расшифрования сообщения используется секретный ключ.

Криптографические системы с открытым ключом в настоящее время широко применяются в различных сетевых протоколах, в частности, в протоколах
{\it TLS\/} и его предшественнике {\it SSL} (лежащих в основе {\it HTTPS}), а так же {\it SSH}, {\it PGP}, {\it S/MIME} и т.\,д.

Рассмотрим случай, когда оправитель хочет послать получателю секретное сообщение.
\begin{enumerate}
   \item Получатель генерирует 2 ключа. Один их них открытый, другой закрытый (секретный). При этом закрытый ключ не должен передаваться по
      открытому каналу, либо его подлинность должна быть гарантирована некоторым сертифицирующим органом.

   \item Отправитель с помощью открытого ключа шифрует сообщение.
   
   \item Получатель с помощью закрытого ключа дешифрует сообщение.

\end{enumerate}

\subsection{Преимущества}
Преимущество асимметричных шифров перед симметричными шифрами состоит в отсутствии необходимости предварительной передачи секретного ключа
по надёжному каналу. Сторона, желающая принимать зашифрованные тексты, в соответствии с используемым алгоритмом вырабатывает пару «открытый
ключ~--- закрытый ключ». Значения ключей связаны между собой, однако вычисление закрытого ключа по открытому должно быть невозможным с
практической точки зрения. Открытый ключ публикуется в открытых справочниках и используется для шифрования информации контрагентоми.
Закрытый ключ держится в секрете и используется для расшифровывания сообщения, переданного владельцу пары ключей.
Для удостоверения аутентичности самих публичных ключей, передаваемых по открытому каналу или получаемых из справочника, обычно
используют сертификаты.

\subsection{Недостатки} 
Асимметричные криптосистемы в чистом виде требуют существенных вычислительных ресурсов, потому на практике используются в сочетании с
другими алгоритмами. Обычно сообщение шифруют временным (сессионным) симметричным ключом, а сам симметричный ключ шифруют ассиметричным.

\subsection{Электронная цифровая подпись}
Для ЭЦП сообщение предварительно подвергается хешированию, а с помощью асиметричного ключа подписывается лишь относительно небольшой
результат хеш-функции.

\section{Алгоритмы RSA и Эль-Гамаля}
\subsection{RSA}
\subsection{Elgamal}
Шифросистема {\bf Эль-Гамаля} {\it (Elgamal)}~--- была предложена в 1984 году. 
В частности стандарты электронной цифровой подписи в США и России базируются именно на ней.
\paragraph{Принцип работы шифросистемы.} 
\begin{enumerate}
   \item Генерируется случайное простое число $p$ длины $k$.
   \item Выбираются случайные числа $x$ и $g$ так, что $1 < x < p$, $1 < g < p$.
   \item Вычисляется $y = g^x \mod p$.
\end{enumerate}

Открытым ключом является тройка $(p,g,y)$, закрытым ключом~---  число $x$.

\paragraph{Шифрование.}
Будем обозначать исходное сообщение $M$.

\begin{enumerate}
   \item Выбирается случайное секретное число $k$, взаимно простое с $p - 1$.
   \item Вычисляется $a = g^k\bmod p$, $b = y^kM\bmod p$, где $M$~--- исходное сообщение.
\end{enumerate}

Пара чисел $(a,b)$ является шифротекстом. При этом длина шифротекста больше длины исходного сообщения $M$ вдвое.

\paragraph{Дешифрование.} 
Зная закрытый ключ $x$, исходное сообщение получается из шифротекста $(a,b)$ по формуле: $M = b / a^x\bmod p.$

Нетрудно проверить, что
$$a^x\equiv g^{kx}\pmod{p}\quad\text{и}\quad\frac{b}{a^x}\equiv \frac{y^kM}{a^x}\equiv \frac{g^{xk}M}{g^{xk}}\equiv M \pmod{p}.$$

\paragraph{Криптостойкость.} 
Криптостойкость данной схемы основана на сложности проблемы дискретного логарифмирования (по известным $p$, $g$ и $y$ приходится искать показатель
степени $x$: $y \equiv g^x \pmod{p}.$


\section{Методы распределение ключей} 
\section{Алгоритмы разделения секрета}

\end{document}
