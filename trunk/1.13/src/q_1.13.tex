\documentclass[a4paper,12pt]{article}

\usepackage[utf8]{inputenc}
\usepackage[T2A]{fontenc}
\usepackage[russian]{babel}

\usepackage{amsmath}
\usepackage{amssymb}

\begin{document}

\section*{Вопрос \No1.13}
{\em Понятие аппроксимации, устойчивости и сходимости численного
решения ДУЧП. Основные методы анализа устойчивости разностных схем
(модельное уравнение, дифференциальное приближение, метод фон Неймана).}


\section{Понятие аппроксимации, устойчивости и сходимости численного
решения ДУЧП}

Пусть имеется область $G\subset\mathbb{R}^p$ с границей $\Gamma$ и
поставлена корректная задача для дифференциального уравнения
с граничными условиями:
\begin{gather*}
Au(x)-f(x)=0,\qquad x\in G,\\
Ru(x)-\mu(x)=0,\qquad x\in\Gamma.
\end{gather*}

Введем в области $G\cup\Gamma$ сетку с шагом $h$, состоящую из множества
внутренних (регулярных) узлов $\omega_h$ и множества граничных (нерегулярных)
узлов $\gamma_h$. Заменим исходную задачу разностным аналогом:
\begin{gather*}
A_hu_h(x)-f_h(x)=0,\qquad x\in\omega_h,\\
R_hu_h(x)-\mu_h(x)=0,\qquad x\in\gamma_h.
\end{gather*}

Близость разностной схемы к исходной задаче будем определять
по величине невязки:
\begin{gather*}
\psi_h(x)=(Au-f)-(A_hu-f_h),\qquad x\in\omega_h,\\
\nu_h(x)=(Ru-\mu)-(R_hu-\mu_h),\qquad x\in\gamma_h.
\end{gather*}

Разностное решение $u_h$ сходится к точному решению $u$, если
$||u_h-u||\to0$ при $h\to0$;
разностное решение имеет порядок точности $p$, если
$||u_h-u||=O(h^p)$ при $h\to0$.

Разностная схема аппроксимирует задачу, если
$||\psi_h||\to0$, $||\nu_h||\to0$, при $h\to0$;
аппроксимация имеет порядок $p$, если
$||\psi_h||=O(h^p)$, $||\nu_h||=O(h^p)$, при $h\to0$.

Разностная схема устойчива, если решение системы разностных уравнений
непрерывно зависит от входных параметров $f$ и $\mu$, причем равномерно
по~$h$.

Разностная схема корректна, если она устойчива и ее решение существует
и единственно при любых $f$ и $\mu$ из некоторого класса функций.

\textbf{Теорема.} Если решение задачи существует, разностная схема
корректна и аппроксимирует задачу на данном решении, то
разностное решение сходится к точному.


\section{Основные методы анализа устойчивости}

TODO

\end{document}

