\documentclass[a4paper,12pt]{article}

\usepackage[utf8]{inputenc}
\usepackage[T2A]{fontenc}
\usepackage[russian]{babel}

\usepackage{amsmath}
\usepackage{amssymb}

\begin{document}

\section*{Вопрос \No1.13}
{\em Понятие аппроксимации, устойчивости и сходимости численного
решения ДУЧП. Основные методы анализа устойчивости разностных схем
(модельное уравнение, дифференциальное приближение, метод фон Неймана).}


\section{Понятие аппроксимации, устойчивости и сходимости численного
решения ДУЧП}

Пусть имеется область $G\subset\mathbb{R}^p$ с границей $\Gamma$ и
поставлена корректная задача для дифференциального уравнения
с граничными условиями:
\begin{gather*}
Au(x)-f(x)=0,\qquad x\in G,\\
Ru(x)-\mu(x)=0,\qquad x\in\Gamma.
\end{gather*}

Введем в области $G\cup\Gamma$ сетку с шагом $h$, состоящую из множества
внутренних (регулярных) узлов $\omega_h$ и множества граничных (нерегулярных)
узлов $\gamma_h$. Заменим исходную задачу разностным аналогом:
\begin{gather*}
A_hu_h(x)-f_h(x)=0,\qquad x\in\omega_h,\\
R_hu_h(x)-\mu_h(x)=0,\qquad x\in\gamma_h.
\end{gather*}

Близость разностной схемы к исходной задаче будем определять
по величине невязки:
\begin{gather*}
\psi_h(x)=(Au-f)-(A_hu-f_h),\qquad x\in\omega_h,\\
\nu_h(x)=(Ru-\mu)-(R_hu-\mu_h),\qquad x\in\gamma_h.
\end{gather*}

Разностное решение $u_h$ сходится к точному решению $u$, если
$||u_h-u||\to0$ при $h\to0$;
разностное решение имеет порядок точности $p$, если
$||u_h-u||=O(h^p)$ при $h\to0$.

Разностная схема аппроксимирует задачу, если
$||\psi_h||\to0$, $||\nu_h||\to0$, при $h\to0$;
аппроксимация имеет порядок $p$, если
$||\psi_h||=O(h^p)$, $||\nu_h||=O(h^p)$, при $h\to0$.

Разностная схема устойчива, если решение системы разностных уравнений
непрерывно зависит от входных параметров $f$ и $\mu$, причем равномерно
по~$h$.

Разностная схема корректна, если она устойчива и ее решение существует
и единств
енно при любых $f$ и $\mu$ из некоторого класса функций.

\textbf{Теорема.} Если решение задачи существует, разностная схема
корректна и аппроксимирует задачу на данном решении, то
разностное решение сходится к точному.


\section{Основные методы анализа устойчивости разностных схем}

\subsection{Модельное уравнение}

Разностные схемы удобно исследовать на простых уравнениях,
точные решения которых известны:
$u'=0\Rightarrow u=C$, $u'=\lambda u\Rightarrow u=u_0e^{\lambda x}$


\subsection{Дифференциальное приближение}

Переходя от точного уравнения $Au-f=0$ к приближенному $A_hu_h-f_h=0$,
мы вносим ошибку, поскольку на самом деле $A_hu_h-f_h=O(h)$.
Оценив $O(h)$, можно получить истинное приближаемое уравнение.

Пример:
$\frac{\partial T}{\partial t}+u\frac{\partial T}{\partial x}=0$

Разностный аналог:
$\frac{T_i^{n+1}-T_i^n}{\Delta t}-u\frac{T_i^n-T_{i-1}^n}{\Delta x}=0$

Разложим $T_i^{n+1}$ и $T_{i-1}^n$ в ряд Тейлора и подставим.
Получилось:
$(\frac{\partial T}{\partial t}+u\frac{\partial T}{\partial x})_i^n
	+0.5(\frac{\partial^2T}{\partial t^2})_i^n\Delta t
    -0.5u(\frac{\partial^2T}{\partial x^2})_i^n\Delta x
    +o(\Delta x,\Delta t)=0$
Это и есть наиболее точно приближаемое уравнение.


\subsection{Метод фон Неймана}

(Он же спектральный признак устойчивости.)

В разностной схеме $A_hu_h-f_h=0$ рассматривается однородное
уравнение $A_hu_h=0$, и каждое вхождения значения функции
в узле сетки $u_{x=x_n}^{t=t_m}$ заменяется на $\lambda^m e^{in\alpha}$.
Из полученного уравнения находится $\lambda$.
Если $|\lambda|\le1$, то схема считается устойчивой.

$\lambda$ "--- множитель перехода гармоники с одного временного
слоя на другой. Условие $|\lambda|\le1$ обеспечивает невозрастание
накопленной ошибки.

\end{document}

