\documentclass[a4paper,12pt]{article}

\usepackage[utf8]{inputenc}
\usepackage[T2A]{fontenc}
\usepackage[russian]{babel}

\usepackage{amsmath}

\begin{document}

\section*{Вопрос \No1.3}
{\em Линейные интегральные уравнения. Связь краевых задач
математической физики с интегральными уравнениями.
Элементы теории потенциала.}


\section{Линейные интегральные уравнения}

Интегральное уравнение "--- уравнение, содержащее неизвестную функцию
под знаком интеграла.

Уравнения Фредгольма:
\begin{align*}
&\text{I рода}&&\int_a^b K(x,t)f(t)dt=g(x),\\
&\text{II рода}&&f(x)-\lambda\int_a^b K(x,t)f(t)dt=g(x).
\end{align*}

$K(x,t)$ "--- ядро интегрального уравнения, $g(x)$ "---
известная функция, $f(x)$ "--- искомая функция.

Если $g(x)=0$, то уравнение называется однородным.

Если $K(x,t)=0$ при $t>x$, получаем уравнение Вольтерра
I и II рода соответственно.

Запишем уравнение II рода в операторном виде: $(I-\lambda K)f=g$.
Если $|\lambda|\sqrt{\int_a^b\int_a^b|K(x,t)|^2dxdt}<1$, то
уравнение решается путем составления ряда:
$f=(\sum_{i=0}^\infty \lambda^i K^i)g$.

Если $K(x,t)=\sum_i g_i(x)h_i(t)$, то ядро называется вырожденным,
и интегральное уравнение сводится к системе линейных уравнений.


Альтернатива Фредгольма (для уравнений 2-го рода): либо неоднородное
уравнение разрешимо при любом $f(x)$, либо однородное уравнение имеет
нетривиальные решения.


\section{Связь краевых задач математической физики
с интегральными уравнениями}

Многие краевые задачи математической физики могут быть сведены
к интегральным уравнениям при помощи функции Грина.

Пусть дано однородное дифференциальное уравнение
$$L[y]=p_0(x)y^{(n)}+p_1(x)y^{(n-1)}+\cdots+p_{n-1}(x)y'+p_n(x)y=0,$$
где функции $p_0,\dots,p_n$ непрерывны на $[a,b]$, $p_0(x)\ne0$ на
$[a,b]$, и краевые условия $V_k(y)=0$, $k=1,\dots,n$, где линейные
формы $V_k$ от $y(a)$, $\dots$, $y^{(n-1)}(a)$, $y(b)$, $\dots$,
$y^{(n-1)}(b)$ являются линейно независимыми.

Функцией Грина этой краевой задачи называется функция $G(x,\xi)$,
построенная для любой точки $a<\xi<b$ и имеющая следующие 4 свойства:
\begin{itemize}
\item $G(x,\xi)$ непрерывна и имеет непрерывные производные по $x$
до $(n-2)$-го порядка включительно при $a\le x\le b$;
\item ее $(n-1)$-я производная имеет разрыв в точке $\xi$:
$$\left.\frac{\partial^{n-1}G(x,\xi)}{\partial x^{n-1}}\right|_{x=\xi+0}
    -\left.\frac{\partial^{n-1}G(x,\xi)}{\partial x^{n-1}}\right|_{x=\xi-0}
    =\frac{1}{p_0(\xi)};$$
\item в каждом из интервалов $[a,\xi)$ и $(\xi,b]$ функция $G(x,\xi)$,
рассматриваемая как функция от $x$, является решением исходного уравнения,
т.е. $L[G]=0$;
\item $G(x,\xi)$ удовлетворяет исходным граничным условиям.
\end{itemize}

\textbf{Теорема.} Если однородная краевая задача имеет лишь тривиальное решение
$y(x)=0$, то оператор $L$ имеет одну и только одну функцию Грина $G(x,\xi)$.

\textbf{Теорема.} Если $G(x,\xi)$ есть функция Грина однородной краевой задачи
$L[y]=0$, $V_k(y)=0$, $k=1,\dots,n$, то решение неоднородной краевой задачи
$L[y]=f(x)$ с теми же граничными условиями дается формулой
$y(x)=\int_{a}^{b}G(x,\xi)f(\xi)d\xi$.

Та же техника позволяет свести нелинейную краевую задачу
или краевую задачу с параметром к интегральному уравнению:
\begin{align*}
L[y]&=f(x,y(x)) \Rightarrow
    &y(x)&=\int_a^b G(x,\xi)f(\xi,y(\xi))d\xi,\\
L[y]&=\lambda y+h(x) \Rightarrow
    &y(x)&=\lambda\int_a^b G(x,\xi)y(\xi)d\xi+\int_a^b G(x,\xi)h(\xi)d\xi.
\end{align*}


\section{Элементы теории потенциала}

\subsection{Объемный потенциал}

$$U=\int_V\rho\frac{1}{r}dV,$$
где $\rho$ "--- плотность потенциала, $r$ "--- расстояние между точкой
где ищется потенциал, и точкой интегрирования.

\textbf{Теорема.} Если плотность потенциала $\rho$ непрерывна,
то $U$ "--- непрерывная и непрерывно дифференцируемая функция.

\textbf{Теорема.} Если плотность потенциала $\rho$ непрерывно дифференцируема,
то $U$ дважды непрерывно дифференцируема и $\Delta U=4\pi\rho$.


\subsection{Потенциалы простого и двойного слоев}

$$V=\int_S\nu\frac{1}{r}dS,\qquad 
  W=\int_S\omega\frac{\partial}{\partial n}\frac{1}{r}dS,$$

\textbf{Теорема.} Если функция $\nu$ непрерывна, то потенциал $V$
непрерывен.

\textbf{Теорема.}
$$
\lim_{M\to M_0}\frac{\partial V}{\partial n}=\left\{\begin{aligned}
&\frac{\partial V(M_0)}{\partial n}+2\pi\nu(M_0), &&\text{$M$ внутри},\\
&\frac{\partial V(M_0)}{\partial n}-2\pi\nu(M_0), &&\text{$M$ снаружи}.
\end{aligned}\right.
$$

\textbf{Теорема.} Если функция $\omega$ непрерывна, то потенциал $W$
непрерывен.

\textbf{Теорема.}
$$
\lim_{M\to M_0}W(M)=\left\{\begin{aligned}
&W(M_0)+2\pi\omega(M_0), &&\text{$M$ снаружи},\\
&W(M_0), &&\text{$M$ на $S$},\\
&W(M_0)-2\pi\omega(M_0), &&\text{$M$ внутри}.
\end{aligned}\right.
$$


\subsection{Применение потенциалов}

$\Delta u|_{x\in G}=0$, $\frac{\partial u}{\partial n}|_{x\in\Gamma}=f$
"--- задача Неймана. Решение ищется в виде потенциала простого слоя,
распространенного по~$\Gamma$.

$\Delta u|_{x\in G}=0$, $u|_{x\in\Gamma}=f$ "--- задача Дирихле.
Решение ищется в виде потенциала двойного слоя, распространенного
по~$\Gamma$.

Т.о., решение уравнения Лапласа сводится к решению интегрального уравнения.

\end{document}

