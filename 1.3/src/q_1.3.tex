\documentclass[a4paper,12pt]{article}

\usepackage[utf8]{inputenc}
\usepackage[T2A]{fontenc}
\usepackage[russian]{babel}

\usepackage{amsmath}

\begin{document}

\section*{Вопрос \No1.3}
{\em Линейные интегральные уравнения. Связь краевых задач
математической физики с интегральными уравнениями.
Элементы теории потенциала.}


\section{Линейные интегральные уравнения}

Интегральное уравнение "--- уравнение, содержащее неизвестную функцию
под знаком интеграла.

Линейное интегральное уравнение 1-го рода:
$$\int_a^b K(x,t)u(t)dt=f(x)$$

Линейное интегральное уравнение 2-го рода (уравнение Фредгольма):
$$u(x)-\lambda\int_a^b K(x,t)u(t)dt=f(x),$$
где $K(x,t)$ "--- ядро интегрального уравнения, $f(x)$ "---
известная функция, $u(t)$ "--- искомая функция.

Если $f(x)=0$, то уравнение называется однородным.

Если $K(x,t)=0$ при $t>x$, получаем уравнение Вольтерра:
$$u(x)-\lambda\int_a^x K(x,t)u(t)dt=f(x).$$

Если $K(x,t)=\sum_i g_i(x)h_i(t)$, то ядро называется вырожденным,
и интегральное уравнение сводится к системе линейных уравнений.


\section{Связь краевых задач математической физики
с интегральными уравнениями}

Многие краевые задачи математической физики могут быть сведены
к интегральным уравнениям при помощи функции Грина.

Пусть дано однородное дифференциальное уравнение
$$L[y]=p_0(x)y^{(n)}+p_1(x)y^{(n-1)}+\cdots+p_{n-1}(x)y'+p_n(x)y=0,$$
где функции $p_0,\dots,p_n$ непрерывны на $[a,b]$, $p_0(x)\ne0$ на
$[a,b]$, и краевые условия $V_k(y)=0$, $k=1,\dots,n$, где линейные
формы $V_k$ от $y(a)$, $\dots$, $y^{(n-1)}(a)$, $y(b)$, $\dots$,
$y^{(n-1)}(b)$ являются линейно независимыми.

Функцией Грина этой краевой задачи называется функция $G(x,\xi)$,
построенная для любой точки $a<\xi<b$ и имеющая следующие 4 свойства:
\begin{itemize}
\item $G(x,\xi)$ непрерывна и имеет непрерывные производные по $x$
до $(n-2)$-го порядка включительно при $a\le x\le b$;
\item ее $(n-1)$-я производная имеет разрыв в точке $\xi$:
$$\left.\frac{\partial^{n-1}G(x,\xi)}{\partial x^{n-1}}\right|_{x=\xi+0}
    -\left.\frac{\partial^{n-1}G(x,\xi)}{\partial x^{n-1}}\right|_{x=\xi-0}
    =\frac{1}{p_0(\xi)};$$
\item в каждом из интервалов $[a,\xi)$ и $(\xi,b]$ функция $G(x,\xi)$,
рассматриваемая как функция от $x$, является решением исходного уравнения,
т.е. $L[G]=0$;
\item $G(x,\xi)$ удовлетворяет исходным граничным условиям.
\end{itemize}

\textbf{Теорема.} Если однородная краевая задача имеет лишь тривиальное решение
$y(x)=0$, то оператор $L$ имеет одну и только одну функцию Грина $G(x,\xi)$.

\textbf{Теорема.} Если $G(x,\xi)$ есть функция Грина однородной краевой задачи
$L[y]=0$, $V_k(y)=0$, $k=1,\dots,n$, то решение неоднородной краевой задачи
$L[y]=f(x)$ с теми же граничными условиями дается формулой
$y(x)=\int_{a}^{b}G(x,\xi)f(\xi)d\xi$.

Та же техника позволяет свести нелинейную краевую задачу
или краевую задачу с параметром к интегральному уравнению:
\begin{align*}
L[y]&=f(x,y(x)) \Rightarrow
    &y(x)&=\int_a^b G(x,\xi)f(\xi,y(\xi))d\xi,\\
L[y]&=\lambda y+h(x) \Rightarrow
    &y(x)&=\lambda\int_a^b G(x,\xi)y(\xi)d\xi+\int_a^b G(x,\xi)h(\xi)d\xi.
\end{align*}


\section{Элементы теории потенциала}

Потенциал "--- понятие, характеризующее широкий класс физических
силовых полей (электрическое, гравитационное и т.п.) и вообще
векторных полей. В общем случае потенциал векторного поля $A(x,y,z)$ "---
скалярная функция $u(x,y,z)$, такая, что $A=\nabla u$. Если такую функцию
можно ввести, то векторное поле $A$ называют потенциальным.

Потенциал векторного поля $A$ определяется не однозначно, а с точностью
до постоянного слагаемого. Поэтому при изучении потенциального поля
представляют интерес лишь разности потенциала в различных точках поля.
Уравнение $u(x,y,z)=c$ представляет поверхность, во всех точках которой
потенциал имеет одинаковую величину; такие поверхности называют
поверхностями уровня, или эквипотенциальными поверхностями.

Переходить от векторного поля к потенциалу и обратно позволяет формула
$\int_V \nabla udV=\int_S u\vec{dS}$. Формула является частным случаем
обобщенной формулы Остроградского-Гаусса
$\int_V \nabla\circ XdV=\int_S X\circ\vec{dS}$, где $X$ "--- скалярное
или векторное поле, $\circ$ "--- обычное, скалярное или векторное умножение.

В теории потенциала рассматривают следующие функции:
\begin{align*}
&\text{ньютонов потенциал точечной массы $m$:}
    &u&=m/r,\\
&\text{ньютонов потенциал массы в объеме $V$:}
    &u&=\int_V\frac{\rho}{r}dV,\\
&\text{потенциал простого слоя:}
    &u&=\int_S\frac{\rho}{r}dS,\\
&\text{потенциал двойного слоя:}
    &u&=\int_S\mu\frac{\partial}{\partial n}\frac{1}{r}dS,\\
%&\text{потенциал бесконечного цилиндра:}
%    &u&=\int_S\ln\frac{1}{r}dS.
\end{align*}

Если $\Delta u=0$ и $u$ непрерывна вместе со своими первыми и вторыми
производными в некоторой области, то $u$ называется гармонической.
$\Delta u$ "--- уравнение Лапласа. Его решение можно искать при помощи
теории потенциалов.

$\Delta u|_{x\in G}=0$, $\frac{\partial u}{\partial n}|_{x\in\Gamma}=f$
"--- задача Неймана. Решение ищется в виде потенциала простого слоя,
распространенного по~$\Gamma$.

$\Delta u|_{x\in G}=0$, $u|_{x\in\Gamma}=f$ "--- задача Дирихле.
Решение ищется в виде потенциала двойного слоя, распространенного
по~$\Gamma$.

\end{document}

